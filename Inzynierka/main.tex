\documentclass[12pt, eng, twoside, openany, final]{mgr}
% \documentclass[eng, printmode]{mgr}

\include{preambula}
% \include{bibliografia}


% re-definiowane polecenie w celu przechowywania nazwiska autora, jego brak powoduje ostrzezenie (Warning) podczas przetwarzania.
\author{Piotr Szczpański} 

% re-definiowane polecenie w celu przechowywania polskiego tytułu pracy magisterskiej, jego brak powoduje ostrzezenie (Warning) podczas przetwarzania.
\title{Moduł sterownika oświetlenia LED} 


% polecenie zdefiniowane w celu przechowywania angielskiego tytułu pracy magisterskiej, jego brak powoduje ostrzezenie (Warning) podczas przetwarzania.
\engtitle{LED lighting controller module} 


% polecenie zdefiniowane w celu przechowywania danych osobowych prowadzacego prace, jego brak powoduje ostrzezenie (Warning) podczas przetwarzania.
\supervisor{dr inż. Antoni Izworski} 

% polecenie zdefiniowane w celu przechowywania danych osobowych opiekuna pracy. W przypadku gdy jest to ta sama osoba co \supervisor, nie nalezy uzywac tego polecenia. Jego brak usunie ze strony tytułowej zbedna informacje.
% \guardian{tytuł, Imie Nazwisko, jednostka} 


% polecenie zdefiniowane w celu przechowywania nazwy kierunku studiów, jego brak powoduje ostrzezenie (Warning) podczas przetwarzania.
\field{Automatyka i Robotyka (AIR)} 


% polecenie zdefiniowane w celu przechowywania nazwy specjalnosci studiów, jego brak powoduje ostrzezenie (Warning) podczas przetwarzania.
\specialisation{Robotyka (ARR)} 

% re-definiowane polecenie w celu przechowywania roku. Standardowo u dołu strony tytułowej wstawiany jest biezacy rok, uzycie tego polecenia pozwala wstawic dowolny rok.
\date{2018}
\usepackage{fancyhdr}

\begin{document}
\maketitle
\tableofcontents
\listoffigures
\newpage

\pagestyle{fancy}
\fancyhead{} 
\fancyfoot{} 
\lhead{Politechnika Wrocławska Wydział Elektroniki}
\rfoot{\thepage}
\lfoot{Szczepański P, Moduł sterownika oświetlenia LED}


\chapter{Wstęp}
\thispagestyle{fancy}
    \section{Wprowadzenie}
    W dzisejszczych czasach oświetlenie w naszych domach opórcz funkcji czysto praktycznej, ma również służyć jako dekoracja i stanowić integralną część wystroju. \\*
    Oświetlenie LED stanowi znaczącą część rynku elektroniki konsumenckiej i w ciągu kilkunastu lat wyparła prawie całkowicie z użytku codziennego wolframowe żarówki i lampy wyładowcze (potocznie zwane świetlówkami). Popularność technologii LED jest spowodowana głównie atrakcyjnymi cenami produktów i znacząco większą sprawnością elektryczną oraz co za tym idzie niższym zużyciem energii elektrycznej.
    
    \section{Cel i zakres pracy}
    Celem pracy jest stworzenie opisu powstawania, dokumentacji użytkownika i dokumentacji technicznej nowoczesnego sterownika oświetlenia LED opartego na platformie mikrokontrolerowej.
    
    Zakres pracy obejmuje opracowanie materiałów produkcyjnych, oprogramowania oraz fizycznego modelu. W szczególności zakres pracy obejmuje: 
    \begin{itemize}
        \item zaprojektowanie elektroniki, 
        \item budowę modułu sterownika oświetlenia,
        \item stworzenie opiu i objaśnienia wykorzystanych technologii,
        \item wytworzenie oprogramowania oraz jego opisu funkcjonalnego,
        \item opracowanie dokumentacji technicznej modułu,
        \item opracowanie instrukcji obsługi modułu oraz dokumentacji dla użytkownika końcowego.
    \end{itemize}
%
\chapter{Założenia projektowe}
\thispagestyle{fancy}  
    Tworzony moduł będzie oferował użytkownikowi możliwość sterowania kolorami i jasnością trójkolorowego paska ledowego oraz sterwoanie dwoma wyjściami przekaźnikowymi. Użytkownik będzie komunikował się z modułem przez czterocalowy dotykowy wyświetlacz. Interfejs będzie udostępniał możliwość ręcznej zmiany stanów podłączonego oświetlenia, programowania własnych sekwencji zmiany kolorów paska LED i stanów wyjść przekaźnikowych oraz czasowego wywoływania zapisanych scen oświetleniowych.\\*
    Gotowy moduł sterowonika oświetlenia LED musi spełniać kryteria jakościowe, w szczególności:
    \begin{itemize}
        \item częstoliwość odświeżania generowango sygnału PWM sterującego pasek ledowy musi być na tyle wysoka, aby nie można było zaobserować migotania,
        
        \item graficzny interfejs użytkownika powinien być przejżysty, intuicyjny w obsłudze oraz na tyle responsywny, żeby
        obsługa modułu była komfortowa,
        
        \item moduł musi działać w ciągłości, tzn. konfiguracja wbudowanego systemu operacyjnego musi zabezpieczać sytuacje
        krytycznych błędów czasu działania, między innymi takich jak: desynchronizacja systemowego zegara, wyłączenie aplikacji interfejsu graficznego
    \end{itemize}
%
\chapter{Opis wykorzystanych technologii}
\thispagestyle{fancy}
    \section{Raspberry Pi}
    Raspberry Pi jest serią jednopłytkowych komputerów opartych na rdzeniach ARM. Pod nazwą Raspberry Pi kryje się również olbżymia społeczność rozwijająca oprogomowanie, tworząca biblioteki i materiały edukacyjne.
    W projekcie wykorzystano model \emph{Raspberry Pi Zero W V1.3} oparty na procesorze  \emph{Broadcom BCM2835 ARM11}. Moduł wyposażony jest m.in. w 512MB pamięci RAM, moduł WiFi, moduł Bluetooth, port miniHDMI. DZIĘKI ?? dużymi możliwościami technicznymi oraz niskiej cenie, modułu ten jest powszechnie wykorzystywany przy prototypowaniu urządzeń elektronicznych IoT.
    \textcolor{red}{(jeszcz coś o linuxie)}
    \section{Język C++}
       \textcolor{red}{zalety względem czystego c , różnice między pythonem}
       Lorem ipsum dolor sit amet, consectetur adipiscing elit. Duis vulputate nulla quis magna iaculis, vitae dignissim turpis malesuada. Nullam lacus eros, fermentum eu tristique at, aliquam ut nibh. Donec in porta dolor. Etiam eu nunc non dui auctor lobortis a eu nunc. Vivamus euismod magna nec nulla auctor porta. Pellentesque semper sapien id mauris eleifend posuere. Pellentesque eget varius justo. Ut molestie nulla mauris, eget pretium quam ultricies in. In posuere tristique nibh.

    Pellentesque libero velit, pellentesque ac vestibulum nec, sollicitudin nec nibh. Curabitur eget lectus dolor. Ut non est vel dui elementum dignissim eget ultricies leo. Fusce sit amet dictum lorem, vel mattis nulla. Duis ut neque semper, egestas purus sit amet, gravida massa. Sed tellus est, rhoncus sed rutrum id, imperdiet non erat. Sed gravida cursus efficitur. Fusce vulputate posuere magna sed rhoncus. Integer convallis ex eu lacus consectetur, eu fermentum urna feugiat. Maecenas et tellus risus. Vivamus eget consequat quam. Vivamus accumsan tempus ante, vitae dignissim odio convallis nec. Nunc molestie sit amet ex pulvinar volutpat.

    Nulla ultricies enim ut auctor viverra. Nulla facilisi. Donec orci sapien, faucibus eu sem eu, bibendum placerat sem. Curabitur condimentum sed leo ac hendrerit. Aliquam sed elementum velit. Donec sagittis felis magna, vel fringilla odio dictum eu. Cras vel accumsan nunc. Etiam dignissim ipsum vel lacus eleifend, sed posuere ante tincidunt. Duis tempor lorem vitae pellentesque lobortis.
    
    \section{Biblioteka QT i QML}
    \textcolor{red}{rys historyczny, krótki opis, zalety QML}
    Fusce sed arcu tristique, pretium quam id, eleifend mi. Vestibulum luctus ex tristique dapibus condimentum. Morbi non fermentum sem. Maecenas sit amet cursus quam, id vestibulum massa. Vivamus vel risus et lacus tempor lobortis. Maecenas aliquet lorem id venenatis euismod. Aliquam leo magna, convallis et lorem quis, semper blandit urna. Proin vel mi at urna scelerisque ultrices iaculis id odio. Suspendisse ornare est mi, sed faucibus ex tristique id. Ut porta efficitur auctor. Ut arcu justo, ultrices non tempor nec, cursus ac nisl. Nulla ac sem erat.

    Duis magna turpis, luctus vel interdum ac, pellentesque dapibus nunc. Aenean mollis libero sapien, et tristique neque interdum quis. Donec ullamcorper lectus nec eros lacinia, sit amet euismod quam venenatis. Orci varius natoque penatibus et magnis dis parturient montes, nascetur ridiculus mus. Nullam porta leo orci, eu tempus ex tristique vitae.
%
\chapter{Dokumentacja użytkownika}
\thispagestyle{fancy}
    \section{Podłączanie zasilania i urządzeń wykonawczych}
    \section{Opis interfejsu graficznego}
    \section{Konfiguracja funkcji}
        \subsection{Tworzenie scen}
        \subsection{Tworzenie animacji}
        \subsection{Tworzenie zdarzeń czasowych}
%
\chapter{Dokumentacja serwisowa}
\thispagestyle{fancy}
    \section{Typowe wykorzystanie}
    Przeznaczeniem modułu jest wykorzystanie w pomieszczeniu użytkowym, do sterowania intensywnością świecenia poszczególnych kolorów paska ledowego oraz do przełączania dwóch dwunastowoltowych wyjść przekaźnikowych. Wyjścia przekaźnikowe zostały umieszczone w projekcie aby pozwolić użytkownikowi podłączyć i sterować np.: stanem jednokolorowego paska LED, stanem żarówki LED lub zewnętrznym przekaźnikiem bądź stycznikiem. Maksymalne parametry elektryczne określone są w sekcji \emph{5.2.1 Elektornika wykonawcza}.
    \newpage
    
    \section{Budowa modułu}
    Sterownik składa się z zaprojektowanej płytki z elektroniką wykonawczą i stabilizatorem napięcia, połączonej wielożyłowym kablem z modułem \emph{Raspberry Pi}, do którego, za pomocą listwy goldpin, podłączony jest wyświetlacz dotykowy.
        \begin{figure}[H]
        \begin{center}
            \includegraphics[width=0.4\textwidth]{diagram.jpg}
            \caption{Schemat ideowy modułu}
        \end{center}
        \end{figure}
        
       
        \begin{figure}[!h]
        	\centering
        	\begin{minipage}[t]{5cm}
        		\centering
        		\includegraphics[scale=0.1]{rpi_goldpin.jpg}
        		\caption{Raspberry Pi z przylutowanymi złączami goldpin}
        	\end{minipage}
        	\hspace{3cm}
        	\begin{minipage}[t]{5cm}
        		\centering
        		\includegraphics[scale=0.1]{rpi_lcd.jpg}
        		\caption{Połączenie Raspberry Pi z wyswietlaczem dotykowym}
        	\end{minipage}
        \end{figure}
        
        \begin{figure}[H]
            \begin{center}
            \includegraphics[width=0.6\textwidth]{poloczona_zoom.jpg}
            \caption{Moduł wykonawczy połączony z Rapsberry Pi}
            \end{center}
        \end{figure}
        
        \newpage
        \subsection{Elektronika wykonawcza}
            \subsubsection{Sterownik paska LED}
                Protoypowanie modułu rozpoczęto od budowy sterownika paska ledowego, który sterowany był z prostego 8-bitowego mikrokontorlera \emph{Atmega 168P}. 
                Sterownik składa się z trzech identycznych kanałów, osobnego dla każdego koloru paska. 
                \begin{figure}[H]
                \begin{center}
                    \includegraphics[width=0.45\textwidth]{sterownik.png}
                    \caption{Schemat trzykanałowego sterownika paska LED}
                \end{center}
                \end{figure}
                Elementem wykonawczym są tranzystory MOSFET (ang. Metal-Oxide Semiconductor Field-Effect Transistor), oznaczone na schemacie symbolami \emph{Q1}, \emph{Q2}, \emph{Q3}. Po podaniu między złącza bramka(ang. gate) i źródło(ang. source), określonego w nocie katalogowej, napięcia, tranzysor zaczyna przewodzić między złączami dren(ang. drain)-źródło. Należy pamiętać, żę między drenem, a źródłem występuje rezystancja określona w nocie katalogowej symbolem Rds(on). Jest to ważny parametr, ponieważ ma on wpływ na ilość energii, która zostanie wytracona w formie ciepła. 
        
                \newpage
        
            \subsubsection{Wyjścia przekaźnikowe}
            Maecenas non massa purus. Sed quis arcu sed lorem maximus luctus quis at ante. Donec porta turpis nec imperdiet posuere. Duis auctor lectus id blandit fermentum. Maecenas sem ex, placerat mollis nulla in, placerat fringilla mauris. Nunc molestie euismod massa, sit amet auctor ligula semper nec. Phasellus metus nisl, venenatis suscipit efficitur in, fringilla ut urna. 
                \begin{figure}[H]
                \begin{center}
                    \includegraphics[width=0.7\textwidth]{przekazniki.png}
                    \caption{Schemat wyjść przekaźnikowych}
                \end{center}
                \end{figure}
        \subsection{Przetwornica napięcia}
        Na płytce znajudje się również przetwornica napięcia, która służy zasileniu Raspberry Pi Zero. Zdecydowano się na wykorzystanie \emph{ON Semiconductor MC34063A} z uwagi na dużą dostępność, niską cenę oraz łatwość użycia. Moduły MC34063 pracują w zakresie napięcia wejściowego od 3V do 40V i mogą być skonfigurowane w trybie obniżania, podwyższania lub odwracania napięcia. Maksymalny prąd wyjściowy to 1.5A, jest to wartość przekraczająca potrzeby przy zasilaniu większości mikroprocesorów. Sterowanie parametrami układu DC-DC odbywa się przez dobór wartości elementów pasywnych. Nota katalogowa układu dostarcza potrzebne w wzory potrzebne do obliczenia wartości elementów.
                \begin{figure}[H]
                \begin{center}
                    \includegraphics[width=0.7\textwidth]{psu.png}
                    \caption{Schemat przetwornicy napięć}
                \end{center}
                \end{figure}
        Parametry wyjściowe układu wynoszą:
        \begin{itemize}
            \item Napięcie wejściowe : 12V
            \item Napięcie wyjściowe : 5V
            \item Częstotliwość      : Hz 
            \item Wahania napięcia wyjściowego : Vpp 
        \end{itemize}
                
        \subsection{Elektronika dodatkowa}
        Na płytce umieszczono również elementy służące wyłącznie weryfikacji poprawności działania układu.
        
        \subsubsection{Przełącznik trybów pracy}
            \begin{figure}[H]
            \begin{center}
                \includegraphics[width=0.4\textwidth]{przelacznik.jpg}
                \caption{Schemat przełącznika trybów pracy}
            \end{center}
            \end{figure}
        W celach zmiany trybów diagnostycznych na płytce znajduje się potrójny przełącznik suwakowy. Z uwagi na 3.3V poziom logiczny Raspberry Pi i fakt, że na płytce znajduje się jednynie przetwornica napięcia o 5-cio V wyjściu należało stowrzyć dzielnik napięciowy tworzony z rezystorów R17 i R18. Rezystory R19, R20, R21 są rezystorami podciągającymi.  
        \subsubsection{Diody sygnalizujące stan pracy}
            \begin{figure}[H]
            \begin{center}
                \includegraphics[width=0.4\textwidth]{wskaznik.jpg}
                \caption{Schemat diod sygnalizacyjnych} 
            \end{center}
            \end{figure}
        Moduł posiada 3 diody sygnalizacyjne służące do informowania użytkownika o błędach lub mogące być wykorzystane przy sprawdzaniu modułu. Rezystory R12, R15, R16 dobrano tak aby intesywność świecenia diod nie sprawawiała dyskonfortu serwisantowi badającemu sterownik.
        
        \subsection{Wyświetlacz dotykowy}
            \begin{figure}[H]
            \begin{center}
                \includegraphics[width=0.5\textwidth]{wyswietlacz.jpg}
                \caption{Wyświetlacz dotykowy używany w module} 
            \end{center}
            \end{figure}
            tft 4cale spi waveshare połączenie z rpi patent
        
        \subsection{Płytka drukowana}
            Po zakończonym etapie prototypowania poszczególnych partii, elementy te zostały połączone w jeden schemat, na podstawie którego zaprojektowano płytkę drukowaną (PCB). 
    \section{Instrukcja montażu}
    Lorem ipsum dolor sit amet, consectetur adipiscing elit. Nam interdum lectus nunc, id aliquet metus tempus quis. Phasellus tincidunt tellus dolor, ut semper est lobortis quis. Quisque imperdiet varius lectus ac tempus. Vivamus molestie, dui in porttitor ornare, quam urna viverra tortor, a placerat diam elit et nibh. Proin ut facilisis nulla, non porttitor massa. Quisque ut aliquam dolor. Donec risus ligula, sagittis sit amet mi ac, venenatis hendrerit tortor. In tincidunt mattis odio nec euismod. Duis vitae hendrerit libero. Nulla bibendum suscipit justo ac tempus. Nunc finibus porta efficitur. Suspendisse sit amet tempor erat. Morbi id posuere justo. Curabitur sed quam sed nisi hendrerit vehicula.

    \section{Sprawdzanie modułu}
            Płytka i opis znajdujący się na niej został zaprojektowany tak, aby w łatwy sposób można było sprawdzić poprawność działania modułu. 
            
            \subsubsection{Sprawdzanie poprawności działania samego modułu wykonawczego}
            Dolna strona PCB zawiera dwa rodzaje oznaczeń, które służą do sprawdzenie poprawności działania elektroniki płytki:
            \begin{itemize}
                \item oznaczaczenia \emph{G*} informujące do którego pinu mikroprosesora podłączone jest dany element modułu, ma to uławtić programowanie i sprawdzanie poprawności konfiguracji pryferiów,
                \item oznaczenia \emph{P*} będące punktami testowymi.
            \end{itemize}
               \begin{figure}[H]
                \begin{center}
                    \includegraphics[width=0.6\textwidth]{pcb_dol_zaz.png}
                    \caption{Punkty kontroli poprawności działania}
                \end{center}
                \end{figure}
            Na płytce znajduje się, również oznaczenie mówiące o poprawnych wartościach między punktami kontrolnymi \emph{P1} - \emph{P4} oraz napięcia, które należy podać między masą a punktami \emph{P5} - \emph{P6}.
            Sprawdzenie poprawności montażu elemtów prztwronicy napięcia, odbywa się przez pomiar napięcia między punktami P1 P2, napięcie to powinnow wynosić 5V z xxx procentową dokładnością.
            Należy również sprawdzić wartość napięcia między punktami P3 P4, które powinno wynosić XXXXXX.
            W celu sprawdzenia sterownika paska LED należy podać 12V kolejno między masą, punktami P5.1, P5.2, P5.3 i mierzyć napięcie na odpowiadających wyjściach sterownika. Moduł działa poprawnie jeśli napięcie na wyjściu każdego z kanałów będzie równe 12V.
            Sprawdzeie wyjść przekaźnikowych odbywa się przez podnie 3.3V, z punktu P4, na punkty P6.1 i P6.2.
            Przekaźnik w momencie przełączania wydaje wyraźny dźwięk (kliknięcie), więc jego brak w chwili podłączenia napięcia oznacza niepoprawne działanie. Kolejnym etapem weryfikacji jest zmierzenie napięcia na wyjściach przekaźnikowych przy zwartych P6.1 lub P6.2 z P4, powinno ono być równe 12V.
            \\* Opisane metody testowania nie wymagają podłączenie Raspberry Pi, ale wymagają podłączenie zasilania 12V. 
            
            \subsubsection{Sprawdzenie poprawności działania całości modułu}
                \begin{figure}[H]
                \begin{center}
                    \includegraphics[width=0.5\textwidth]{dip_zoom.jpg}
                    \caption{Przełącznik trybów i diody sygnalizacyjne}
                \end{center}
                \end{figure}
            Sterownik udostępnia 3 tryby serwisowe wybierana za pomocą przełącznika suwakowego.
            Oprogramowanie sterownika odczytuje wartości, wejść do których podłączony jest przełącznik, w momencie startu systemu, więc aby włączyć tryb serwisowy należy odłączyć zasilianie modułu, wybrać tryb i dopiero podłączyć zasilanie. Nie jest dozwolnoe przełączenie dwóch lub więcej przełączników, w takim przypadku system zignoruje te informacje i uruchomi moduł w trybie pracy normalnej. Przejście modułu w tryb serwisowy sygnalizowane jest ciągłym podświetleniem diody \emph{LED3}.
            
            Dostępne tryby:
            \begin{enumerate}
                \item przełącznik w pozycji 1 - w trybie tym sterownik będzie włączał i wyłączał z odstępem 1 sekundy wyjścia modułu w kolejnośc : kanał czerwony, kanał zielony, kanał niebieski, wyjście przekaźnikowe 1, wyjście przekaźnikowe 2. Tryb ten pozwala sprawdzić poprawność działania elektroniki wyjściowej oraz podłączenia modułu Raspberry Pi.
                
                \item przełącznik w pozycji 2 - tryb służy do sprawdzenia poprawności działania wyświetlacza, czyli poprawności wyświelania kolorów, czasu odświeżania dokładności powierzchni dotykowej.
                Na ekranie wyświeli się licznik inkrementujący swoją warość co 200ms, napis "TEST" w 3 rozmiarach oraz tło aplikacji będzie zmieniało kolory z częstotliwością 500ms w kolejności: czerwony, zielony, niebieski, biały. Ponadto na ekranie widoczne będą 4 pola dotykowe, których naciśnięcie będzie aktywowało diodę \emph{LED2}, w przypadku wykrycia dotyku poza, którymś z tych pól załączona zostanie dioda \emph{LED1}. 
                
                \item przełącznik w pozycji 3 - tryb służący do kontroli obciążenia procesora systemu, zajętości RAM-u oraz temperatury na procesorze. Informacje te będą wyświetlane na ekranie. Ponadto zostanie wyświelony plik kontrolny z poprzedniego uruchomienia modułu.
            \end{enumerate}
                
            \newpage
            
            \section{Dokumentacja oprogramowania}
                \begin{figure}[H]
                \begin{center}
                    \includegraphics[width=0.5\textwidth]{inz_diag.png}
                    \caption{Schemat przepływu informacji w oprogramowaniu w trybie normalnego użytku}
                \end{center}
                \end{figure}
                
                \subsection{Interfejs graficzny}
                \subsection{}
                \subsection{Opis scenariuszy programowych}
                \subsection{Wykorzystane biblioteki}
                \subsection{}
            
\chapter{Wnioski i uwagi}
\thispagestyle{fancy}
  \cite{CleanCode,EffectiveModern,RpiBeginner}


% \listoffigures
% \listoftables
\bibliographystyle{plabbrv}

\addcontentsline{toc}{chapter}{Bibliografia} %utworzenie w spisie treści pozycji Bibliografia
\bibliography{bibliografia} % wstawia bibliografię korzystając z pliku bibliografia.bib - dotyczy BibTeXa, jeżeli nie korzystamy z BibTeXa należy użyć otoczenia thebibliography



\end{document}